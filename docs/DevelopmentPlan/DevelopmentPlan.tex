\documentclass{article}

\usepackage{booktabs}
\usepackage{tabularx}
\usepackage{hyperref} % For hyperlinks in the PDF
\usepackage{longtable}

\usepackage[margin=1in]{geometry}


\title{Development Plan\\\progname}

\author{\authname}

\date{\today}

%% Comments

\usepackage{color}

\newif\ifcomments\commentstrue %displays comments
%\newif\ifcomments\commentsfalse %so that comments do not display

\ifcomments
\newcommand{\authornote}[3]{\textcolor{#1}{[#3 ---#2]}}
\newcommand{\todo}[1]{\textcolor{red}{[TODO: #1]}}
\else
\newcommand{\authornote}[3]{}
\newcommand{\todo}[1]{}
\fi

\newcommand{\wss}[1]{\authornote{magenta}{SS}{#1}} 
\newcommand{\plt}[1]{\authornote{cyan}{TPLT}{#1}} %For explanation of the template
\newcommand{\an}[1]{\authornote{cyan}{Author}{#1}}
 % For comments
%% Common Parts

\renewcommand{\progname}{
    ProgName}
\renewcommand{\authname}{
    Team \#, Team Name\\
    Student 1 name\\
    Student 2 name\\
    Student 3 name\\
    Student 4 name}
\usepackage{hyperref}
    \hypersetup{colorlinks=true, linkcolor=blue, citecolor=blue, filecolor=blue,
                urlcolor=blue, unicode=false}
    \urlstyle{same}
                                
 % For common macros

\begin{document}

\maketitle

\begin{table}[hp]
\caption{Revision History} \label{TblRevisionHistory}
\begin{tabularx}{\textwidth}{llX}
\toprule
\textbf{Date} & \textbf{Developer(s)} & \textbf{Change}\\
\midrule
2025-09-18 & Zongcheng Li   & Initial draft of Development Plan \\
2025-09-18 & All Members    & Check-in with TA and updates based on feedback \\
2025-09-20 & Zongcheng Li   & Finish remaining sections of the Development Plan \\
2025-09-21 & Ji Zhang       & Updated Appendix, adjusted formatting, and added a reference.\\
2025-09-21 & Changhao Wu    & Revised Team Meeting Plan section and Team Member Roles section \\
2025-09-22 & Changhao Wu    & Revised Main Risks and Mitigation Strategies section with clearer risks and more detailed mitigation steps \\
2025-09-22 & Changhao Wu    & Revised Project Decomposition and Scheduling section \\
2025-09-22 & Changhao Wu    & Revised Team Communication Plan section and Project Decomposition and Scheduling\\
... & ... & ...\\

\bottomrule
\end{tabularx}
\end{table}

\newpage{}

%\wss{Put your introductory blurb here.  Often the blurb is a brief roadmap of
%what is contained in the report.}

\section*{Development Plan Overview}
The development plan for \progname{} outlines how our team will organize, develop, 
and deliver an AI-based system to identify cows by their black-white patterns on their back 
and detect potential health issues. This document includes our Confidential information, 
IP to protect, Copyright license, Team meeting plan, Team communication plan, Team member 
roles, Workflow plan, Project decomposition and scheduling, Proof of concept demonstration 
plan, Expected technology, and Coding standard.

% \wss{Additional information on the development plan can be found in the
% \href{https://gitlab.cas.mcmaster.ca/courses/capstone/-/blob/main/Lectures/L02b_POCAndDevPlan/POCAndDevPlan.pdf?ref_type=heads}
% {lecture slides}.}

\section{Confidential Information?}

%\wss{State whether your project has confidential information from industry, or
%not.  If there is confidential information, point to the agreement you have in
%place.}
%\wss{For most teams this section will just state that there is no confidential
%information to protect.}

Our project does not involve confidential information from industry. 
All data and models will be based on publicly available datasets or 
collected in compliance with open-source or academic research guidelines.

\section{IP to Protect}

%\wss{State whether there is IP to protect.  If there is, point to the agreement.
%All students who are working on a project that requires an IP agreement are also
%required to sign the ``Intellectual Property Guide Acknowledgement.''}

There is no proprietary intellectual property (IP) to protect for this project. 
The outputs will be published under an open-source license to encourage 
collaboration and reproducibility.

NEED UPDATE LATER, since we haven't discuss the details with the industry partner yet.

\section{Copyright License}

%\wss{What copyright license is your team adopting.  Point to the license in your
%repo.}

The team has chosen the MIT License, which will be included in the GitHub 
repository. This license provides flexibility for others to use, modify, 
and distribute the project with proper attribution.


\section{Team Meeting Plan}

%\wss{How often will you meet? where?}

%\wss{If the meeting is a physical location (not virtual), out of an abundance of
%caution for safety reasons you shouldn't put the location online}

%\wss{How often will you meet with your industry advisor?  when?  where?}

%\wss{Will meetings be virtual?  At least some meetings should likely be
%in-person.}

%\wss{How will the meetings be structured?  There should be a chair for all meetings.  There should be an agenda for all meetings.}

The team will meet every Tuesday and Sunday between 6:00--10:00 PM, with each meeting lasting approximately 30 minutes. 
Meetings will be held online using Microsoft Teams or Discord. If an in-person session is required for testing, integration, or major 
decision-making, a separate time and location will be arranged in advance. These twice-weekly meetings do not include 
the mandatory weekly meeting with the TA during the Thursday tutorial time, nor the separate weekly meeting with the supervisor.

Each meeting will be structured as follows:

\begin{enumerate}
    \item The agenda for each meeting will be posted as a GitHub issue ahead of time.
    \item Each team member will provide updates on their assigned tasks (progress, challenges).
    \item The team will discuss ongoing issues, distribute new tasks, and set short-term goals.
    \item Each meeting will have a designated chair and a note-taker. All meeting outcomes will be recorded and shared with the team.
\end{enumerate}

In addition:
\begin{itemize}
    \item All members must attend the weekly TA meeting during the Thursday tutorial time.
    \item The supervisor will hold a weekly virtual meeting with the team for progress updates, feedback, and Q\&A.
\end{itemize}





\section{Team Communication Plan}

%\wss{Issues on GitHub should be part of your communication plan.}
\begin{itemize}
  \item \textbf{Discord / WeChat}: Primary platforms for internal team communication, informal discussions, quick updates, and project-related meetings.
  \item \textbf{Discord / MS Teams}: Used for meetings with supervisor and TA.
  \item \textbf{GitHub}: Code-related discussions, issue tracking, project management, and documentation.
  \item \textbf{Email}: Formal communication and sharing of important documents.
  \item \textbf{MS Teams (Shared Folder)}: Collaborative document editing and storage.
\end{itemize}


\section{Team Member Roles}

% \wss{You should identify the types of roles you anticipate, like notetaker,
% leader, meeting chair, reviewer.  Assigning specific people to those roles is
% not necessary at this stage.  In a student team the role of the individuals will
% likely change throughout the year.}

Every team member is responsible for attending meetings, responding to messages, creating issues, coding, testing, 
reviewing, and documentation. In addition, members will take on the following roles:

\begin{itemize}
  \item \textbf{Project Manager}: Organizes the workflow, monitors progress toward deadlines, balances workload among members, and communicates project status to the supervisor.
  
  \item \textbf{Meeting Chair}: Prepares agendas, leads meetings, facilitates discussions, ensures decisions are made efficiently, and keeps meetings on track.
  
  \item \textbf{Notetaker}: Team members will take turns serving as notetaker, responsible for recording meeting minutes, capturing action items, and summarizing key discussions to be shared with the group.
  
  \item \textbf{Quality Assurance}: All team members will participate in reviewing deliverables to ensure completeness, accuracy, and adherence to project requirements. The review process will be shared collectively rather than assigned to a single individual.
\end{itemize}

\noindent
Roles are expected to be flexible, and members may rotate or assume different responsibilities as project needs evolve, ensuring both accountability and adaptability.

\section{Workflow Plan}

% \begin{itemize}
% 	\item How will you be using git, including branches, pull request, etc.?
% 	\item How will you be managing issues, including template issues, issue
% 	classification, etc.?
%   \item Use of CI/CD
% \end{itemize}
The following workflow will be adopted for the project:
\begin{enumerate}
  \item Pull any changes from the main branch before starting new work.
	\item Create a new branch for each feature or bug fix, following the naming convention `feature/feature-name` or `bugfix/bug-name`.
	\item Create a draft of the structure of the code/documentation.
	\item Implement the functions without dependencies first, then the functions with dependencies.
	\item Commit changes frequently with clear, descriptive messages.
	\item Push the branch to the remote repository on GitHub.
	\item Open a pull request (PR) to merge the feature/bugfix branch into the main branch.
	\item Assign at least one team member as a reviewer for the PR.
	\item Address any feedback from the reviewer(s) and make necessary changes.
	\item Once approved, the PR will be merged into the main branch.
	\item Delete the feature/bugfix branch after merging to keep the repository clean.
	\item Use GitHub Issues to track tasks, bugs, and feature requests, with appropriate labels for classification.
	\item Update any changes in the weekly check-in meetings.
\end{enumerate}


\section{Project Decomposition and Scheduling}

% \begin{itemize}
%   \item How will you be using GitHub projects?
%   \item Include a link to your GitHub project
% \end{itemize}

\begin{itemize}
  \item Link to GitHub Project: \url{https://github.com/InfantMob/Cow-puter-Vision}
  \item The project will be decomposed into the following major components:
    \begin{itemize}
      \item Requirement Analysis and Specification
      \item System Design and Architecture
      \item Data Collection, Preprocessing, and Augmentation
      \item External Component Integration, and Adaptive Modification
      \item Neural Network Model Construction and Training
      \item Model Evaluation, Validation, and Improvement
      \item Pipeline Integration and Deployment
      \item Preparation of Final Deliverables
    \end{itemize}
  \item Each component is further broken down into smaller tasks, which are tracked using GitHub Issues and linked to the Kanban board.
  \item The project is managed using GitHub Projects with a Kanban board setup. 
        The board contains the following columns: 
        \begin{itemize}
          \item No Status / Backlog
          \item Ready
          \item In Progress
          \item In Review
          \item Done
        \end{itemize}  \item Each task will be assigned to a team member, with due dates set based on the overall project timeline.
  \item Regular updates will be made during weekly meetings to ensure progress is on track.
\end{itemize}

%table
\begin{tabularx}{\textwidth}{@{}X r@{}}
    \toprule
    \textbf{Deliverable}                            & \textbf{Date}       \\
    \midrule
    Problem Statement, POC Plan, Development Plan   & Sep.\ 22,\ 2025     \\
    Req. Doc. and Hazard Analysis Revision 0        & Oct.\ 6,\ 2025      \\
    V\&V Plan Revision 0                            & Oct.\ 27, 2025      \\
    Design Document Revision -1                     & Nov.\ 10, 2025      \\
    Proof of Concept Demonstration                  & Nov.\ 17-28, 2025   \\
    Term Break                                      & ---                    \\
    Design Document Revision 0                      & Jan.\ 19, 2026      \\
    Revision 0 Demonstration                        & Feb.\ 2-13, 2026    \\
    V\&V Report and Extras Revision 0               & Mar.\ 9, 2026       \\
    Final Demonstration (Revision 1)                & Mar.\ 23-29, 2026   \\
    EXPO Demonstration                              & TBD                 \\
    Final Documentation (Revision 1)                & Apr.\ 6, 2026       \\
    \bottomrule
\end{tabularx}


% \wss{How will the project be scheduled?  This is the big picture schedule, not
% details. You will need to reproduce information that is in the course outline
% for deadlines.}

% What is the main risk, or risks, for the success of your project?  What will you
% demonstrate during your proof of concept demonstration to convince yourself that
% you will be able to overcome this risk?
\section{Proof-of-Concept Demonstration Plan}

The proof-of-concept (POC) is designed to directly address the key risks to project success by demonstrating that the core functionality can be achieved under realistic constraints.

\subsection*{Main Risks and Mitigation Strategies}
\begin{enumerate}
  \item \textbf{Feasibility of the computer vision pipeline}:
      The pipeline for this task consists of multiple components (e.g., detection, tracking, and individual identification). 
      If all components were to be implemented from scratch, the project would likely exceed the available time 
      and result in significantly lower accuracy compared to state-of-the-art methods. Since the overall pipeline accuracy 
      is bounded by the weakest component, unreliable submodules would drag down the entire system’s performance.  
      \begin{itemize}
        \item \emph{Mitigation}: 
            \item[]Use established models as the foundation for certain components, such as YOLO for cow detection 
              and a tracking library (e.g., ByteTrack or DeepSORT) for maintaining identities across frames. 
              This allows the team to focus its efforts on the novel and most critical task---\emph{individual cow identification}. 
              This division of effort makes the project feasible while ensuring the final pipeline maintains competitive accuracy.
      \end{itemize}

  \item \textbf{Pipeline integration complexity}:
      Even if individual components (detection, tracking, and identification) work well independently, integrating them into a seamless 
      pipeline may introduce unexpected errors, latency, or data mismatches between modules.  
      \begin{itemize}
        \item \emph{Mitigation}: 
          \item[] Use standardized data formats (e.g., consistent bounding box representations, unified frame rates) to minimize compatibility issues between modules.  
          \item[] Employ logging and modular testing to monitor intermediate outputs at each stage, so problems can be traced without debugging the entire pipeline.  
      \end{itemize}

  \item \textbf{Availability and quality of training data}:
        Datasets may be insufficient in size or diversity, leading to poor generalization.  
        \begin{itemize}
          \item \emph{Mitigation}: 
          \item[] Combine multiple public datasets (e.g., OpenCows2020, MultiCamCows2024). Perform data augmentation and, if necessary, collect a small supplemental dataset.
        \end{itemize}

  \item \textbf{Data domain mismatch (training vs. real-world deployment)}:
      Public datasets may differ significantly from real farm environments in terms of lighting, camera angles, and resolution. 
      As a result, a model trained only on public datasets may underperform when deployed in the field.  
      \begin{itemize}
        \item \emph{Mitigation}: 
        \item[] Incorporate “non-ideal” data samples in the POC, such as videos captured under different lighting 
        or viewing angles, to test robustness. 
      \end{itemize}

  \item \textbf{Tracking robustness in crowded or occluded scenes}:
      When multiple cows overlap or partially occlude each other in the video feed, off-the-shelf tracking components may lose track 
      of identities.  
      \begin{itemize}
        \item \emph{Mitigation}: 
        \item[] Evaluate multiple tracking libraries (e.g., ByteTrack, DeepSORT) during the POC to identify which 
        performs best in crowded scenarios. 
      \end{itemize}

\end{enumerate}

\subsection*{Demonstration Plan}
The POC will demonstrate a minimal but complete pipeline that directly addresses the identified risks:  
\begin{enumerate}
  \item \textbf{Cow detection and tracking}: Run YOLO for cow detection and integrate a tracking library 
        (e.g., ByteTrack or DeepSORT) to maintain identities across frames.  
        This step verifies that off-the-shelf components can provide a reliable foundation 
        (\emph{addresses pipeline feasibility and tracking robustness}).  

  \item \textbf{Individual identification prototype}: Apply the team’s custom model to assign stable IDs 
        to detected cows.  
        This tests whether individual recognition can function reliably given outputs from detection and tracking 
        (\emph{addresses feasibility and data quality}).  

  \item \textbf{Domain robustness check}: Use a small set of “non-ideal” video samples with varied lighting 
        and camera angles to observe performance differences.  
        (\emph{addresses domain mismatch}).  

  \item \textbf{Integration validation}: Combine detection, tracking, and identification into a single pipeline, 
        while logging intermediate results at each stage.  
        This confirms that the modules can be integrated without major latency or data mismatch 
        (\emph{addresses integration complexity}).  

  \item \textbf{User-facing output}: Display results in a simple UI showing bounding boxes and correct labels 
        across frames stably. This demonstrates the pipeline’s functionality and makes evaluation of performance straightforward.  
\end{enumerate}

If successful, the POC will provide evidence that each major risk can be managed and that the overall project is feasible within the available timeframe.


\section{Expected Technology}


% \wss{What programming language or languages do you expect to use?  What external
% libraries?  What frameworks?  What technologies.  Are there major components of
% the implementation that you expect you will implement, despite the existence of
% libraries that provide the required functionality.  For projects with machine
% learning, will you use pre-trained models, or be training your own model?  }

% \wss{The implementation decisions can, and likely will, change over the course
% of the project.  The initial documentation should be written in an abstract way;
% it should be agnostic of the implementation choices, unless the implementation
% choices are project constraints.  However, recording our initial thoughts on
% implementation helps understand the challenge level and feasibility of a
% project.  It may also help with early identification of areas where project
% members will need to augment their training.}

% Topics to discuss include the following:

% \begin{itemize}
% \item Specific programming language
% \item Specific libraries
% \item Pre-trained models
% \item Specific linter tool (if appropriate)
% \item Specific unit testing framework
% \item Investigation of code coverage measuring tools
% \item Specific plans for Continuous Integration (CI), or an explanation that CI
%   is not being done
% \item Specific performance measuring tools (like Valgrind), if
%   appropriate
% \item Tools you will likely be using?
% \end{itemize}

The team plans to use the following technologies for the development of \progname{}:
\begin{itemize}
\item Programming language: Python
\item External Libraries: TensorFlow, OpenCV, NumPy, Flask
\item Pre-trained models: YOLOv5 for object detection
\item Linter tool: flake8
\item Unit testing framework: pytest
\item Investigation of code coverage measuring tools: coverage.py
\item Plans for Continuous Integration (CI), or an explanation that CI
  is not being done: GitHub Actions will be used for CI/CD.
\item Performance measuring tools: TensorBoard for monitoring model training.
\item Tools: Git, GitHub, and GitHub Projects.
\end{itemize}

% \wss{git, GitHub and GitHub projects should be part of your technology.}

\section{Coding Standard}

% \wss{What coding standard will you adopt?}
Because we want to ensure code readability and maintainability,
we will adopt the PEP 8 coding standard for Python, which emphasizes readability and consistency.

To be specific, we will follow these key guidelines:
\begin{itemize}
  \item Use 4 spaces per indentation level.
  \item Limit all lines to a maximum of 79 characters.
  \item Use blank lines to separate functions and classes, and larger blocks of code inside functions.
  \item Use docstrings to describe all public modules, functions, classes, and methods.
  \item Use meaningful variable and function names that convey their purpose.
  \item Follow naming conventions: use CamelCase for classes, snake\_case for functions and variables.
  \item Avoid using global variables; prefer passing parameters and returning values.
  \item Write comments to explain the purpose of complex code sections.
  \item Ensure consistent use of whitespace in expressions and statements.
  \item Use version control (Git) effectively, with clear commit messages and branching strategies.
  \item Conduct regular code reviews to ensure adherence to the coding standard and improve code quality.
  \item Use automated tools like flake8 to check for compliance with the coding standard.
\end{itemize}

Additionally, we encourage team members to write comments and provide constructive feedback during code reviews
to continuously improve our coding practices.

\newpage{}

\section*{Appendix --- Reflection}

% \wss{Not required for CAS 741}

The purpose of reflection questions is to give you a chance to assess your own
learning and that of your group as a whole, and to find ways to improve in the
future. Reflection is an important part of the learning process.  Reflection is
also an essential component of a successful software development process.  

Reflections are most interesting and useful when they're honest, even if the
stories they tell are imperfect. You will be marked based on your depth of
thought and analysis, and not based on the content of the reflections
themselves. Thus, for full marks we encourage you to answer openly and honestly
and to avoid simply writing ``what you think the evaluator wants to hear.''

Please answer the following questions.  Some questions can be answered on the
team level, but where appropriate, each team member should write their own
response:
 

\begin{enumerate}
    \item Why is it important to create a development plan prior to starting the
    project?

     As a group, we all strongly agree that creating a development plan before starting a project is essential because it provides the team with a clear \textquotedblleft roadmap.\textquotedblright\ With a well-structured plan, the team can:
\begin{itemize}
    \item Align goals and avoid ambiguous responsibilities;
    \item Identify potential risks and challenges early on, reducing conflicts later;
    \item Break down the project into manageable tasks with realistic deadlines;
    \item Allocate resources effectively and establish clear communication channels, ensuring that everyone works toward the same objective.
\end{itemize}
    \item In your opinion, what are the advantages and disadvantages of using
    CI/CD?

 As a group, we share the same opinion on the usage of CI/CD. It has many advantages in our project,but also some disadvantages.
\subsection*{Advantages}
\begin{itemize}
    \item \textbf{Faster development:} Code can be integrated and tested more frequently, allowing issues to be detected early and preventing large-scale bug accumulation.
    \item \textbf{Reliable deployment:} Automated pipelines reduce human error, making deployment faster and more stable.
    \item \textbf{Better collaboration:} CI/CD provides a shared platform for the team, improving collaboration and feedback speed.
    \item \textbf{Efficient iteration:} Continuous integration and delivery accelerate product iteration and help reduce technical debt.
\end{itemize}

\subsection*{Disadvantages}
\begin{itemize}
    \item \textbf{Difficult to start:} Setting up CI/CD pipelines requires significant time, effort, and sometimes infrastructure costs.
    \item \textbf{Maintenance overhead:} As projects grow, pipelines become increasingly complex, requiring ongoing optimization and maintenance.
    \item \textbf{Over-reliance risk:} Heavy automation may cause developers to rely too much on the system and overlook details.
    \item \textbf{Configuration risks:} If pipelines are not properly configured, frequent errors may occur, slowing down development instead of speeding it up.
\end{itemize}
    \item What disagreements did your group have in this deliverable, if any,
    and how did you resolve them?

        One of the main disagreements in our group was about how detailed our development plan should be. Some members wanted a high-level overview to maintain flexibility, while others preferred a very detailed, task-by-task breakdown. We resolved this by combining both approaches: we created a high-level road map but also added more details for the first sprint, leaving later stages more flexible.

\end{enumerate}




\section*{Appendix --- Team Charter}

% \wss{borrows from
% \href{https://engineering.up.edu/industry_partnerships/files/team-charter.pdf}
% {University of Portland Team Charter}}
%This link doesn't work

\subsection*{External Goals}

% \wss{What are your team's external goals for this project? These are not the
% goals related to the functionality or quality fo the project.  These are the
% goals on what the team wishes to achieve with the project.  Potential goals are
% to win a prize at the Capstone EXPO, or to have something to talk about in
% interviews, or to get an A+, etc.}
Our team's external goals for this project include:
\begin{itemize}
  \item Make something useful for the industry partner.
  \item Gain practical experience in AI/ML development and write on resume.
  \item Achieve a high grade in the course, we aim for an A.
\end{itemize}

\subsection*{Attendance}

\subsubsection*{Expectations}

% \wss{What are your team's expectations regarding meeting attendance (being on
% time, leaving early, missing meetings, etc.)?}
We expect all team members to attend all scheduled meetings on time and stay for the entire duration. \\
We understand that emergencies may arise, and in such cases, team members should notify the team as soon as possible.\\
Missing meetings without a acceptable excuse may result in consequences as outlined in the "Stay on Track" section.

\subsubsection*{Acceptable Excuse}

% \wss{What constitutes an acceptable excuse for missing a meeting or a deadline?
% What types of excuses will not be considered acceptable?}
We accept excuses such as:
\begin{itemize}
  \item Illness or medical emergencies
  \item Family emergencies
  \item Academic commitments (e.g., exams, presentations, course conflicts)
  \item Pre-approved absences (e.g., prior commitments, work obligations)
\end{itemize}

\subsubsection*{In Case of Emergency}

% \wss{What process will team members follow if they have an emergency and cannot
% attend a team meeting or complete their individual work promised for a team
% deliverable?}
Actions to take in case of an emergency:
\begin{itemize}
  \item Notify the team as soon as possible via MS Teams or email or other communication channels.
  \item If it is a meeting with TA or industry partner, notify them as well.
  \item Provide a brief explanation of the emergency and expected duration of absence.
  \item If possible, delegate tasks to other team members to ensure continuity.
  \item Upon return, catch up on missed work and communicate with the team about any challenges faced during the absence.
\end{itemize}

\subsection*{Accountability and Teamwork}

\subsubsection*{Quality} 

% \wss{What are your team's expectations regarding the quality
% of team members' preparation for team meetings and the quality of the
% deliverables that members bring to the team?}
Expectations regarding quality:
\begin{itemize}
  \item Prepare questions, updates, and relevant materials before each meeting.
  \item Deliverables should be completed on time, and be able to run.
  \item Team members should actively participate in discussions, and replyig to messages in a timely manner.
\end{itemize}

\subsubsection*{Attitude}

% \wss{What are your team's expectations regarding team members' ideas,
% interactions with the team, cooperation, attitudes, and anything else regarding
% team member contributions?  Do you want to introduce a code of conduct?  Do you
% want a conflict resolution plan?  Can adopt existing codes of conduct.}

Our expectations regarding attitude:
\begin{itemize}
  \item Be respectful of each other's ideas and opinions, even if they differ from our own.
  \item Be open to constructive feedback and willing to provide the same to others.
  \item Notify the team in advance (at least 24 hours) if unable to meet deadlines or attend meetings, this is important.
  \item Keep up to date with project progress and contribute actively to discussions and tasks.
  \item At least check the messages on MS Teams or Discord once a day.
\end{itemize}

\subsubsection*{Stay on Track}

% \wss{What methods will be used to keep the team on track? How will your team
% ensure that members contribute as expected to the team and that the team
% performs as expected? How will your team reward members who do well and manage
% members whose performance is below expectations?  What are the consequences for
% someone not contributing their fair share?}

% \wss{You may wish to use the project management metrics collected for the TA and
% instructor for this.}

% \wss{You can set target metrics for attendance, commits, etc.  What are the
% consequences if someone doesn't hit their targets?  Do they need to bring the
% coffee to the next team meeting?  Does the team need to make an appointment with
% their TA, or the instructor?  Are there incentives for reaching targets early?}
To keep the team on track, we will set clear milestones and deadlines for each phase of the project.
Announcements and reminders will be sent on MS Teams and Discord channels
to ensure everyone is aware of upcoming tasks and deadlines.

Attendance, code commits, and task completion will be monitored. 
If a team member consistently fails to meet expectations, the team will first discuss the issue privately 
with the individual to understand any challenges they may be facing. If the issue persists, 
the team may consider meeting with the TA or instructor to seek guidance on how to best support the individual.

\subsubsection*{Team Building}

% \wss{How will you build team cohesion (fun time, group rituals, etc.)? }
Our team will build cohesion through regular check-ins, 
celebrating milestones, and organizing occasional social 
activities outside of meetings to foster camaraderie.

We encourage open communication and provide constructive feedback so that 
each member gains experience in teamwork.

\subsubsection*{Decision Making} 

% \wss{How will you make decisions in your group? Consensus?  Vote? How will you
% handle disagreements? }

When there is a disagreement, the team will first discuss each member's opinion, 
and then vote if necessary. The majority opinion will be adopted.
If we cannot reach a consensus, we will seek advice from our supervisor or TA.


\newpage{}

\section*{References}

\begin{enumerate}
  \item Marinache, A. \textit{Proof of Concept and Development Plan (Lecture Slides)}. McMaster University, CAS 741.  
  Available at: \url{https://gitlab.cas.mcmaster.ca/courses/capstone/-/blob/main/Lectures/L02b_POCAndDevPlan/POCAndDevPlan.pdf}

  \item University of Portland. \textit{Team Charter Guidelines}.  
  Available at: \url{https://engineering.up.edu/industry_partnerships/files/team-charter.pdf}

  \item Python Software Foundation. \textit{PEP 8 - Style Guide for Python Code}.  
  Available at: \url{https://peps.python.org/pep-0008/}

  \item GitHub Docs. \textit{About GitHub Actions CI/CD}.  
  Available at: \url{https://github.com/solutions/use-case/ci-cd}
\end{enumerate}

\end{document}

