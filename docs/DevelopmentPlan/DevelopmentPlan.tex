\documentclass{article}

\usepackage{booktabs}
\usepackage{tabularx}
\usepackage{hyperref} % For hyperlinks in the PDF
\usepackage{longtable}

\title{Development Plan\\\progname}

\author{\authname} 

\date{\today}

%% Comments

\usepackage{color}

\newif\ifcomments\commentstrue %displays comments
%\newif\ifcomments\commentsfalse %so that comments do not display

\ifcomments
\newcommand{\authornote}[3]{\textcolor{#1}{[#3 ---#2]}}
\newcommand{\todo}[1]{\textcolor{red}{[TODO: #1]}}
\else
\newcommand{\authornote}[3]{}
\newcommand{\todo}[1]{}
\fi

\newcommand{\wss}[1]{\authornote{magenta}{SS}{#1}} 
\newcommand{\plt}[1]{\authornote{cyan}{TPLT}{#1}} %For explanation of the template
\newcommand{\an}[1]{\authornote{cyan}{Author}{#1}}
 % For comments
%% Common Parts

\renewcommand{\progname}{
    ProgName}
\renewcommand{\authname}{
    Team \#, Team Name\\
    Student 1 name\\
    Student 2 name\\
    Student 3 name\\
    Student 4 name}
\usepackage{hyperref}
    \hypersetup{colorlinks=true, linkcolor=blue, citecolor=blue, filecolor=blue,
                urlcolor=blue, unicode=false}
    \urlstyle{same}
                                
 % For common macros

\begin{document}

\maketitle

\begin{table}[hp]
\caption{Revision History} \label{TblRevisionHistory}
\begin{tabularx}{\textwidth}{llX}
\toprule
\textbf{Date} & \textbf{Developer(s)} & \textbf{Change}\\
\midrule
2025-09-18 & Zongcheng Li & Initial draft of Development Plan \\
2025-09-18 & All Members& Check-in with TA and updates based on feedback \\
2025-09-20 & Zongcheng Li & Finish remaining sections of the Development Plan \\
2025-09-20 & Baoning Zhang & Finish reflection \\
... & ... & ...\\
\bottomrule
\end{tabularx}
\end{table}

\newpage{}

%\wss{Put your introductory blurb here.  Often the blurb is a brief roadmap of
%what is contained in the report.}

\section{Development Plan Overview}
The development plan for \progname{} outlines how our team will organize, develop, 
nd deliver an AI-based system to identify cows by their black-white patterns on their back 
and detect potential health issues. This document includes our Confidential information, 
IP to protect, Copyright license, Team meeting plan, Team communication plan, Team member 
roles, Workflow plan, Project decomposition and scheduling, Proof of concept demonstration 
plan, Expected technology, and Coding standard.

\wss{Additional information on the development plan can be found in the
\href{https://gitlab.cas.mcmaster.ca/courses/capstone/-/blob/main/Lectures/L02b_POCAndDevPlan/POCAndDevPlan.pdf?ref_type=heads}
{lecture slides}.}

\section{Confidential Information?}

%\wss{State whether your project has confidential information from industry, or
%not.  If there is confidential information, point to the agreement you have in
%place.}
%\wss{For most teams this section will just state that there is no confidential
%information to protect.}

Our project does not involve confidential information from industry. 
All data and models will be based on publicly available datasets or 
collected in compliance with open-source or academic research guidelines.

\section{IP to Protect}

%\wss{State whether there is IP to protect.  If there is, point to the agreement.
%All students who are working on a project that requires an IP agreement are also
%required to sign the ``Intellectual Property Guide Acknowledgement.''}

There is no proprietary intellectual property (IP) to protect for this project. 
The outputs will be published under an open-source license to encourage 
collaboration and reproducibility.

NEED UPDATE LATER, since we haven't discuss the details with the industry partner yet.

\section{Copyright License}

%\wss{What copyright license is your team adopting.  Point to the license in your
%repo.}

The team has chosen the MIT License, which will be included in the GitHub 
repository. This license provides flexibility for others to use, modify, 
and distribute the project with proper attribution.


\section{Team Meeting Plan}

%\wss{How often will you meet? where?}

%\wss{If the meeting is a physical location (not virtual), out of an abundance of
%caution for safety reasons you shouldn't put the location online}

%\wss{How often will you meet with your industry advisor?  when?  where?}

%\wss{Will meetings be virtual?  At least some meetings should likely be
%in-person.}

%\wss{How will the meetings be structured?  There should be a chair for all meetings.  There should be an agenda for all meetings.}
\begin{itemize}
  \item Frequency: Team check-in meetings will be held twice weekly via Microsoft Teams or Zoom, all team members are expected to attend
  and share their progress.
  \item In-Person: An in-person meeting will be scheduled at least once per month on campus.
  \item The industry advisor will meet with the team weekly, virtually. Including check-in and Q\&A.
  \item All members must attend the weekly meeting with TA, usually on Thursday tutorial time.
  \item Each meeting will have a designated chair and note-taker, with an agenda shared in advance.
\end{itemize}

\section{Team Communication Plan}

%\wss{Issues on GitHub should be part of your communication plan.}
\begin{itemize}
  \item Primary communication will be through MS Teams for quick messages and updates.
  \item Email will be used for formal communications and sharing important documents.
  \item Regular progress will be shared in a dedicated GitHub repository.
  \item GitHub Issues will be used to track lecture attendance, tasks, bugs, and feature requests.
  \item A shared file folder on MS Teams will be used for collaborative document editing and storage.
\end{itemize}

\section{Team Member Roles}

% \wss{You should identify the types of roles you anticipate, like notetaker,
% leader, meeting chair, reviewer.  Assigning specific people to those roles is
% not necessary at this stage.  In a student team the role of the individuals will
% likely change throughout the year.}

Every team member are responsible for attending meetings, replying to messages, creating issues, coding, testing,
reviewing code, and documentation.  In addition, team members will take on
the following rotating roles:

\subsection*{Meeting Chair}
Leads meetings and ensures meetings stay on track. \\
Every team member will take turns being the chair.

\subsection*{Note-Taker}
Records meeting minutes, captures action items, and distributes notes to the team. \\
Every team member will take turns being the note-taker.

\subsection*{Reviewer}
Reviews code contributions, provides feedback, and ensures coding standards are met. \\
Zongcheng Li, Ji Zhang and Baoning Zhang will primarily take on this role, but all team members are encouraged to participate in code reviews.

\subsection*{Integrator}
Merges code contributions, resolves conflicts, and ensures the codebase remains stable. \\
Changhao Wu and Zongcheng Li will primarily take on this role, but all team members are encouraged to participate in integration tasks.

\subsection*{Lead Developer}
Oversees the technical direction of the AI/ML pipeline, coordinates integration 
between modules, and supports other members in debugging and implementation. \\
Changhao Wu and Zongcheng Li will primarily take on this role, but all team members are encouraged to contribute to technical decisions. \\
All team members will contribute to development tasks.

\subsection*{Communication Lead}
Prepare agendas, manages communication with the industry advisor, schedules meetings, and ensures follow-up on action items. \\
Zongcheng Li and Changhao Wu will primarily take on this role, but all team members are encouraged to participate in communication tasks.

\section{Workflow Plan}

% \begin{itemize}
% 	\item How will you be using git, including branches, pull request, etc.?
% 	\item How will you be managing issues, including template issues, issue
% 	classification, etc.?
%   \item Use of CI/CD
% \end{itemize}
The following workflow will be adopted for the project:
\begin{enumerate}
  \item Pull any changes from the main branch before starting new work.
	\item Create a new branch for each feature or bug fix, following the naming convention `feature/feature-name` or `bugfix/bug-name`.
	\item Create a draft of the structure of the code/documentation.
	\item Implement the functions without dependencies first, then the functions with dependencies.
	\item Commit changes frequently with clear, descriptive messages.
	\item Push the branch to the remote repository on GitHub.
	\item Open a pull request (PR) to merge the feature/bugfix branch into the main branch.
	\item Assign at least one team member as a reviewer for the PR.
	\item Address any feedback from the reviewer(s) and make necessary changes.
	\item Once approved, the PR will be merged into the main branch.
	\item Delete the feature/bugfix branch after merging to keep the repository clean.
	\item Use GitHub Issues to track tasks, bugs, and feature requests, with appropriate labels for classification.
	\item Update any changes in the weekly check-in meetings.
\end{enumerate}


\section{Project Decomposition and Scheduling}

% \begin{itemize}
%   \item How will you be using GitHub projects?
%   \item Include a link to your GitHub project
% \end{itemize}

\begin{itemize}
  \item Link to GitHub Project: \url{https://github.com/InfantMob/Cow-puter-Vision}
  \item The project will be decomposed into the following major components:
    \begin{itemize}
      \item Requirement Analysis and Specification
      \item System Design and Architecture
      \item Data Collection and Preprocessing 
      \item Model Selection and Training
      \item Model Evaluation and Validation
      \item Deployment and Integration
      \item Preparation of Final Deliverables
    \end{itemize}
  \item Each component will be further broken down into smaller tasks, which will be tracked using GitHub Issues.
  \item GitHub Projects will be used to manage the workflow, with columns for "To Do", "In Progress", "In Review", and "Done".
  \item Each task will be assigned to a team member, with due dates set based on the overall project timeline.
  \item Regular updates will be made during weekly meetings to ensure progress is on track.
\end{itemize}

%table
\noindent \begin{longtable}{ p{9.7cm} r}
  \caption{Project Deliverables and Weights} \label{TblDeliverablesWeights}\\
   \textbf{Deliverable} & \textbf{Deadline}\\
   \midrule
   \endhead
   Team Formed, Project Selected & Week 3 \\
   
   Problem Statement, POC Plan, Development Plan &
   Week 4 \\

   Req.\ Doc.\ and Hazard Analysis Revision 0 &
   week 6\\

   V\&V Plan Revision 0 & week 8\\

   Design Document Revision -1 & week 10\\

   Proof of Concept Demonstration & week 11 and 12\\
   Term Break \\
   Design Document Revision 0 & week 16\\

   Revision 0 Demonstration & week 18 and 19\\

   V\&V Report and Extras Revision 0 & week 22\\

   Final Demonstration (Revision 1) & week 24\\

   EXPO Demonstration & week 26\\

   Final Documentation (Revision 1)\newline 
    - Problem Statement\newline
    - Development Plan\newline
    - Proof of Concept (POC) Plan\newline
    - Requirements Document\newline
    - Hazard Analysis\newline
    - Design Document\newline
    - V\&V Plan\newline
    - V\&V Report\newline
    - Extra Documentation 1\newline
    - Extra Documentation 2\newline
    - Source Code\newline & week 26\\

\end{longtable}



% \wss{How will the project be scheduled?  This is the big picture schedule, not
% details. You will need to reproduce information that is in the course outline
% for deadlines.}

\section{Proof of Concept Demonstration Plan}

% What is the main risk, or risks, for the success of your project?  What will you
% demonstrate during your proof of concept demonstration to convince yourself that
% you will be able to overcome this risk?
\subsection*{Main Risks}
The main risks identified for the success of the project include:
\begin{itemize}
  \item Technical feasibility of the AI algorithms
  \item Availability and quality of training data
  \item Integration with end users' computing environments
  \item Time constraints and resource limitations
  \item User acceptance and feedback
\end{itemize}

\subsection*{Demonstration Plan}
The proof of concept demonstration will focus on showcasing the core 
functionality of the AI-based cow identification. \\
This will include:
\begin{itemize}
  \item A working prototype that can identify cows from images or video feeds.
  \item A basic user interface for interacting with the system.
  \item Initial results from model training and evaluation.
  \item A plan for addressing the identified risks, including data collection strategies, 
  algorithm selection, and integration approaches.
\end{itemize}
If the proof of concept is successful, it will provide confidence that the project can be completed within the given constraints.

\section{Expected Technology}


% \wss{What programming language or languages do you expect to use?  What external
% libraries?  What frameworks?  What technologies.  Are there major components of
% the implementation that you expect you will implement, despite the existence of
% libraries that provide the required functionality.  For projects with machine
% learning, will you use pre-trained models, or be training your own model?  }

% \wss{The implementation decisions can, and likely will, change over the course
% of the project.  The initial documentation should be written in an abstract way;
% it should be agnostic of the implementation choices, unless the implementation
% choices are project constraints.  However, recording our initial thoughts on
% implementation helps understand the challenge level and feasibility of a
% project.  It may also help with early identification of areas where project
% members will need to augment their training.}

% Topics to discuss include the following:

% \begin{itemize}
% \item Specific programming language
% \item Specific libraries
% \item Pre-trained models
% \item Specific linter tool (if appropriate)
% \item Specific unit testing framework
% \item Investigation of code coverage measuring tools
% \item Specific plans for Continuous Integration (CI), or an explanation that CI
%   is not being done
% \item Specific performance measuring tools (like Valgrind), if
%   appropriate
% \item Tools you will likely be using?
% \end{itemize}

The team plans to use the following technologies for the development of \progname{}:
\begin{itemize}
\item Programming language: Python
\item External Libraries: TensorFlow, OpenCV, NumPy, Flask
\item Pre-trained models: YOLOv5 for object detection
\item Linter tool: flake8
\item Unit testing framework: pytest
\item Investigation of code coverage measuring tools: coverage.py
\item Plans for Continuous Integration (CI), or an explanation that CI
  is not being done: GitHub Actions will be used for CI/CD.
\item Performance measuring tools: TensorBoard for monitoring model training.
\item Tools: Git, GitHub, and GitHub Projects.
\end{itemize}

% \wss{git, GitHub and GitHub projects should be part of your technology.}

\section{Coding Standard}

% \wss{What coding standard will you adopt?}
Because we want to ensure code readability and maintainability,
we will adopt the PEP 8 coding standard for Python, which emphasizes readability and consistency.

To be specific, we will follow these key guidelines:
\begin{itemize}
  \item Use 4 spaces per indentation level.
  \item Limit all lines to a maximum of 79 characters.
  \item Use blank lines to separate functions and classes, and larger blocks of code inside functions.
  \item Use docstrings to describe all public modules, functions, classes, and methods.
  \item Use meaningful variable and function names that convey their purpose.
  \item Follow naming conventions: use CamelCase for classes, snake\_case for functions and variables.
  \item Avoid using global variables; prefer passing parameters and returning values.
  \item Write comments to explain the purpose of complex code sections.
  \item Ensure consistent use of whitespace in expressions and statements.
  \item Use version control (Git) effectively, with clear commit messages and branching strategies.
  \item Conduct regular code reviews to ensure adherence to the coding standard and improve code quality.
  \item Use automated tools like flake8 to check for compliance with the coding standard.
\end{itemize}

Additionally, we encourage team members to write comments and provide constructive feedback during code reviews
to continuously improve our coding practices.

\newpage{}

\section*{Appendix --- Reflection}

\wss{Not required for CAS 741}

The purpose of reflection questions is to give you a chance to assess your own
learning and that of your group as a whole, and to find ways to improve in the
future. Reflection is an important part of the learning process.  Reflection is
also an essential component of a successful software development process.  

Reflections are most interesting and useful when they're honest, even if the
stories they tell are imperfect. You will be marked based on your depth of
thought and analysis, and not based on the content of the reflections
themselves. Thus, for full marks we encourage you to answer openly and honestly
and to avoid simply writing ``what you think the evaluator wants to hear.''

Please answer the following questions.  Some questions can be answered on the
team level, but where appropriate, each team member should write their own
response:
 

\begin{enumerate}
    \item Why is it important to create a development plan prior to starting the
    project?

    \item In your opinion, what are the advantages and disadvantages of using
    CI/CD?

    \item What disagreements did your group have in this deliverable, if any,
    and how did you resolve them?

\end{enumerate}

Zongcheng's answer: \\
\begin{enumerate}
  \item It is important to create a development plan prior to starting the project because it shows clear direction of development, 
helps identify potential risks and challenges early on and avoid potential conflicts. \\
We had a discussion on the development plan and 
get some important consensus, including how the tasks will be assigned, how the communication will be, and how the workflow will be.
So that everyone is on the same page and work towards the same goal. It also helps us to understand the deadlines so that nobody will 
miss any deliverables.
  \item From my perspective, using CI/CD has several advantages/disadvantages:
  \begin{itemize}
    \item Advantages:
      \begin{itemize}
        \item Faster development: it is a rapid way to deliver code changes, allowing for quicker feedback and iteration.
        \item Better collaboration: provides a shared platform, good for team collaboration.
        \item Minimize error: since it is automated, it reduces the likelihood of manual error in repeated tasks.
      \end{itemize}
    \item Disadvantages:
      \begin{itemize}
        \item Difficult to start: it is complex and time-consuming to set up initially.
        \item Maintenance overhead: it needs continuous maintenance to ensure it works properly.
        \item Make me  lazier: since it is automated, I may rely too much on it and not pay attention to the details.
      \end{itemize}
  \end{itemize}
\end{enumerate}

Baoning's answer: \\
\begin{enumerate}
  \item It is essential because it gives the team a clear road map. If we do not have such a development plan, there might be scope creep, unclear responsibilities, misaligned expectations occurring. With a clear plan, we are capable to break down the project into manageable tasks, set realistic deadlines, and allocate resources effectively. It not only increase efficiency, but also decreases possibility of potential conflict.
  \item 
  \begin{itemize}
    \item Advantage: Using CI/CD ensures that code is integrated and tested frequently, which reduces the risk of major bugs piling up. Automated pipelines also make deployment faster and more reliable, improving overall team productivity. Otherwise, technical debt would be heavier and heavier.

    \item Disadvantage: Setting up CI/CD requires a lot of initial effort, and sometimes significant infrastructure costs. If they are not properly configured, pipelines may break frequently, slowing down development instead of speeding it up. As projects grow, pipelines become more complex, thus, build times increase, this would require maintainence and optimization.
\end{itemize}
	\item One of the main disagreements in our group was about how detailed our development plan should be. Some members wanted a high-level overview to maintain flexibility, while others preferred a very detailed, task-by-task breakdown. We resolved this by combining both approaches: we created a high-level road map but also added more details for the first sprint, leaving later stages more flexible.
\end{enumerate}




\section*{Appendix --- Team Charter}

\wss{borrows from
\href{https://engineering.up.edu/industry_partnerships/files/team-charter.pdf}
{University of Portland Team Charter}}
%This link doesn't work

\subsection*{External Goals}

% \wss{What are your team's external goals for this project? These are not the
% goals related to the functionality or quality fo the project.  These are the
% goals on what the team wishes to achieve with the project.  Potential goals are
% to win a prize at the Capstone EXPO, or to have something to talk about in
% interviews, or to get an A+, etc.}
Our team's external goals for this project include:
\begin{itemize}
  \item Make something useful for the industry partner.
  \item Gain practical experience in AI/ML development and write on resume.
  \item Achieve a high grade in the course, we aim for an A.
\end{itemize}

\subsection*{Attendance}

\subsubsection*{Expectations}

% \wss{What are your team's expectations regarding meeting attendance (being on
% time, leaving early, missing meetings, etc.)?}
We expect all team members to attend all scheduled meetings on time and stay for the entire duration. \\
We understand that emergencies may arise, and in such cases, team members should notify the team as soon as possible.\\
Missing meetings without a acceptable excuse may result in consequences as outlined in the "Stay on Track" section.

\subsubsection*{Acceptable Excuse}

% \wss{What constitutes an acceptable excuse for missing a meeting or a deadline?
% What types of excuses will not be considered acceptable?}
We accept excuses such as:
\begin{itemize}
  \item Illness or medical emergencies
  \item Family emergencies
  \item Academic commitments (e.g., exams, presentations, course conflicts)
  \item Pre-approved absences (e.g., prior commitments, work obligations)
\end{itemize}

\subsubsection*{In Case of Emergency}

% \wss{What process will team members follow if they have an emergency and cannot
% attend a team meeting or complete their individual work promised for a team
% deliverable?}
Actions to take in case of an emergency:
\begin{itemize}
  \item Notify the team as soon as possible via MS Teams or email or other communication channels.
  \item If it is a meeting with TA or industry partner, notify them as well.
  \item Provide a brief explanation of the emergency and expected duration of absence.
  \item If possible, delegate tasks to other team members to ensure continuity.
  \item Upon return, catch up on missed work and communicate with the team about any challenges faced during the absence.
\end{itemize}

\subsection*{Accountability and Teamwork}

\subsubsection*{Quality} 

% \wss{What are your team's expectations regarding the quality
% of team members' preparation for team meetings and the quality of the
% deliverables that members bring to the team?}
Expectations regarding quality:
\begin{itemize}
  \item Prepare questions, updates, and relevant materials before each meeting.
  \item Deliverables should be completed on time, and be able to run.
  \item Team members should actively participate in discussions, and replying to messages in a timely manner.
\end{itemize}

\subsubsection*{Attitude}

% \wss{What are your team's expectations regarding team members' ideas,
% interactions with the team, cooperation, attitudes, and anything else regarding
% team member contributions?  Do you want to introduce a code of conduct?  Do you
% want a conflict resolution plan?  Can adopt existing codes of conduct.}

Our expectations regarding attitude:
\begin{itemize}
  \item Be respectful of each other's ideas and opinions, even if they differ from our own.
  \item Be open to constructive feedback and willing to provide the same to others.
  \item Notify the team in advance (at least 24 hours) if unable to meet deadlines or attend meetings, this is important.
  \item Keep up to date with project progress and contribute actively to discussions and tasks.
  \item At least check the messages on MS Teams or Discord once a day.
\end{itemize}

\subsubsection*{Stay on Track}

% \wss{What methods will be used to keep the team on track? How will your team
% ensure that members contribute as expected to the team and that the team
% performs as expected? How will your team reward members who do well and manage
% members whose performance is below expectations?  What are the consequences for
% someone not contributing their fair share?}

% \wss{You may wish to use the project management metrics collected for the TA and
% instructor for this.}

% \wss{You can set target metrics for attendance, commits, etc.  What are the
% consequences if someone doesn't hit their targets?  Do they need to bring the
% coffee to the next team meeting?  Does the team need to make an appointment with
% their TA, or the instructor?  Are there incentives for reaching targets early?}
To keep the team on track, we will set clear milestones and deadlines for each phase of the project.
Announcements and reminders will be sent on MS Teams and Discord channels
to ensure everyone is aware of upcoming tasks and deadlines.

Attendance, code commits, and task completion will be monitored. 
If a team member consistently fails to meet expectations, the team will first discuss the issue privately 
with the individual to understand any challenges they may be facing. If the issue persists, 
the team may consider meeting with the TA or instructor to seek guidance on how to best support the individual.

\subsubsection*{Team Building}

% \wss{How will you build team cohesion (fun time, group rituals, etc.)? }
Our team will build cohesion through regular check-ins, 
celebrating milestones, and organizing occasional social 
activities outside of meetings to foster camaraderie.

We encourage open communication and provide constructive feedback so that 
each member gains experience in teamwork.

\subsubsection*{Decision Making} 

% \wss{How will you make decisions in your group? Consensus?  Vote? How will you
% handle disagreements? }

When there is a disagreement, the team will first discuss each member's opinion, 
and then vote if necessary. The majority opinion will be adopted.
If we cannot reach a consensus, we will seek advice from our industry advisor or TA.


\end{document}
