\documentclass{article}
\usepackage{parskip} 
\usepackage{tabularx}
\usepackage{booktabs}

\title{Problem Statement and Goals\\\progname}

\author{\authname}

\date{}

%% Comments

\usepackage{color}

\newif\ifcomments\commentstrue %displays comments
%\newif\ifcomments\commentsfalse %so that comments do not display

\ifcomments
\newcommand{\authornote}[3]{\textcolor{#1}{[#3 ---#2]}}
\newcommand{\todo}[1]{\textcolor{red}{[TODO: #1]}}
\else
\newcommand{\authornote}[3]{}
\newcommand{\todo}[1]{}
\fi

\newcommand{\wss}[1]{\authornote{magenta}{SS}{#1}} 
\newcommand{\plt}[1]{\authornote{cyan}{TPLT}{#1}} %For explanation of the template
\newcommand{\an}[1]{\authornote{cyan}{Author}{#1}}

%% Common Parts

\renewcommand{\progname}{
    ProgName}
\renewcommand{\authname}{
    Team \#, Team Name\\
    Student 1 name\\
    Student 2 name\\
    Student 3 name\\
    Student 4 name}
\usepackage{hyperref}
    \hypersetup{colorlinks=true, linkcolor=blue, citecolor=blue, filecolor=blue,
                urlcolor=blue, unicode=false}
    \urlstyle{same}
                                


\begin{document}

\maketitle

\begin{table}[hp]
    \caption{Revision History} \label{TblRevisionHistory}
        \begin{tabularx}{\textwidth}{llX}
        \toprule
        \textbf{Date} & \textbf{Developer(s)} & \textbf{Change}\\
        \midrule
            Sept. 16, 2025  & Changhao Wu           & Initial Draft         \\
            Date            & Name(s)               & Description of changes\\
            ...             & ...                   & ...                   \\
        \bottomrule
        \end{tabularx}
\end{table}

\section{Problem Statement}

\wss{You should check your problem statement with the
\href{https://github.com/smiths/capTemplate/blob/main/docs/Checklists/ProbState-Checklist.pdf}
{problem statement checklist}.} 

\wss{You can change the section headings, as long as you include the required
information.}


\subsection{Problem}

    On modern dairy farms, many routine tasks require farmers to quickly locate and 
    separate specific cows from the herd. Examples include providing medical treatment 
    to a sick cow, monitoring the recovery of an injured animal, collecting milk samples 
    for quality testing, administering vaccines, or tracking cows with unusual feeding 
    or reproductive behaviors.  

    \setlength{\parskip}{0.6em} 

    Currently, locating specific animals is highly time-consuming. Farmers often 
    have to walk through an entire pen, visually check ear tag numbers one by one, 
    and manually identify the right cow. This process is not only slow and 
    labor-intensive, but also stressful for both workers and animals.

    \setlength{\parskip}{0.6em} 

    Alternative identification methods such as RFID chips or electronic collars 
    do exist, but they require costly infrastructure, can be prone to technical 
    failure, and may still require manual intervention to physically separate or 
    observe the animal. In addition, ear tags can be lost or damaged, further reducing 
    reliability.  

    \setlength{\parskip}{0.6em} 

    The problem is compounded by the fact that cows are large, mobile herd animals. 
    In a crowded barn or pasture, they are constantly moving, blocking each other 
    from view, or clustering together, making manual identification even more 
    difficult. Under these conditions, locating a single cow can take far longer 
    than planned and disrupt other farm operations. 


\subsection{Inputs and Outputs}


\begin{itemize}
    \item \textbf{Inputs}
          \begin{itemize}
              \item Video footage of dairy cows from static or mobile cameras
          \item Optional annotated datasets consisting of static images or pre-recorded videos (training and evaluation)
          \end{itemize}

    \item \textbf{Outputs}
          \begin{itemize}
              \item Bounding boxes around each detected cow
              \item Tracking IDs assigned to each cow across video frames
              \item Predicted identity of each cow (based on coat pattern classification)
              \item Annotated video stream showing IDs and identities
          \end{itemize}
\end{itemize}


\wss{Characterize the problem in terms of ``high level'' inputs and outputs.  
Use abstraction so that you can avoid details.}


\subsection{Stakeholders}

\subsubsection*{\color{blue}{Direct Stakeholders}}
\begin{enumerate}
    \item\textbf{Cattleytics Inc. (Industry Partner)}: The primary stakeholder 
          and end user of the system. They benefit directly from improved efficiency 
          in locating, identifying, and monitoring individual cows for treatment, 
          feeding management, and general herd health.
    \end{enumerate}


\subsubsection*{\color{blue}{Indirect Stakeholders}}
\begin{enumerate}
    \item \textbf{Veterinarians}: By enabling faster and more reliable identification of target animals, 
            veterinarians can administer treatments and health checks more efficiently, improving overall herd health management.
    \item \textbf{Other Dairy Farmers and Ranchers}: Since the system is designed with relatively low deployment cost, it can be adopted by other farms to reduce 
            labor demands, increase operational efficiency, and ultimately improve profitabilit.
    \item \textbf{Animal Welfare and Protection Organizations}: Better tracking and monitoring ensure that cows receive 
            timely care, reducing stress and discomfort. These outcomes align with the goals of organizations advocating 
            for animal well-being and ethical treatment in agriculture.
\end{enumerate}


\subsection{Environment}

\subsubsection*{\color{blue}{Hardware Environment}}
\begin{itemize}
    \item Deployment will take place in dairy farm environments such as barns, feeding areas, or open pens, 
          where lighting, shadows, and occlusion from other animals may affect visibility.
    \item Input devices will consist of fixed surveillance cameras installed in barns or portable cameras (e.g., smartphones).
    \item Development and training will primarily be conducted on a GPU-enabled workstation 
      (e.g., a desktop equipped with an NVIDIA RTX 4060 or higher-class GPU).
    \item For later testing, the system may also be evaluated on lightweight hardware (e.g., NVIDIA Jetson) 
          to explore potential edge deployment, but the primary focus is on farm-level camera input rather than embedded control.
\end{itemize}


\subsubsection*{\color{blue}{Software Environment}}
\begin{itemize}
    \item Operating System: Ubuntu Linux (WSL2 on Windows will also be used for development).
    \item Programming Language: Python.
    \item Core Libraries and Frameworks: PyTorch, Ultralytics YOLOv8, OpenCV, scikit-learn, and ByteTrack.
    \item Development Tools: Git (for version control), GitHub (for remote collaboration and repository hosting), 
      GitHub Projects (for task management), GitHub Actions (for CI/CD), and VS Code as the primary IDE.
    \item Testing and Evaluation Tools: pytest for unit testing, flake8/black for linting and formatting, 
          and TensorBoard/Matplotlib for model evaluation and visualization.
\end{itemize}

\wss{Hardware and Software Environment}


\section{Goals}
The primary goal of this project is to design and implement a computer vision system that can:
    \begin{enumerate}
        \item Detect and localize cows in video streams.
        \item Track individual cows across consecutive frames.
        \item Identify cows based on unique coat patterns, enabling individual-level monitoring.
    \end{enumerate}

\section{Stretch Goals}
    \begin{itemize}
        \item Deploy the system on an embedded device (e.g., Jetson Nano/Xavier).
        \item Extend recognition to other animals or farm contexts.
        \item Provide a user-friendly dashboard for farmers to visualize and export cow activity/identity logs.
    \end{itemize}


\section{Extras}
    \subsection{Performance Report:}
        A comprehensive performance report will be created to document the system's 
        accuracy, speed, and reliability. This will include quantitative evaluation of detection, tracking, 
        and identification models under different conditions (e.g., lighting, occlusion, herd density). 
        The report will provide insights into system limitations and suggest areas for future improvement.
    \subsection{User Manual:}
        A user-friendly manual will be prepared to guide non-technical users, such as farmers 
        and dairy workers, through system setup and operation. The manual will include step-by-step installation 
        instructions, examples of typical use cases, troubleshooting tips, and recommendations for best practices 
        when deploying the system on a farm.



\wss{For CAS 741: State whether the project is a research project. This
designation, with the approval (or request) of the instructor, can be modified
over the course of the term.}

\wss{For SE Capstone: List your extras.  Potential extras include usability
testing, code walkthroughs, user documentation, formal proof, GenderMag
personas, Design Thinking, etc.  (The full list is on the course outline and in
Lecture 02.) Normally the number of extras will be two.  Approval of the extras
will be part of the discussion with the instructor for approving the project.
The extras, with the approval (or request) of the instructor, can be modified
over the course of the term.}

\newpage{}

\section*{Appendix --- Reflection}

\wss{Not required for CAS 741}

The purpose of reflection questions is to give you a chance to assess your own
learning and that of your group as a whole, and to find ways to improve in the
future. Reflection is an important part of the learning process.  Reflection is
also an essential component of a successful software development process.  

Reflections are most interesting and useful when they're honest, even if the
stories they tell are imperfect. You will be marked based on your depth of
thought and analysis, and not based on the content of the reflections
themselves. Thus, for full marks we encourage you to answer openly and honestly
and to avoid simply writing ``what you think the evaluator wants to hear.''

Please answer the following questions.  Some questions can be answered on the
team level, but where appropriate, each team member should write their own
response:


\begin{enumerate}
    \item What went well while writing this deliverable? 
    \item What pain points did you experience during this deliverable, and how
    did you resolve them?
    \item How did you and your team adjust the scope of your goals to ensure
    they are suitable for a Capstone project (not overly ambitious but also of
    appropriate complexity for a senior design project)?
\end{enumerate}  

\end{document}