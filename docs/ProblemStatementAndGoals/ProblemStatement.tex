\documentclass{article}
\usepackage{parskip} 
\usepackage{tabularx}
\usepackage{booktabs}
\usepackage[margin=1in]{geometry}

\title{Problem Statement and Goals\\\progname}

\author{\authname}

\date{}

%% Comments

\usepackage{color}

\newif\ifcomments\commentstrue %displays comments
%\newif\ifcomments\commentsfalse %so that comments do not display

\ifcomments
\newcommand{\authornote}[3]{\textcolor{#1}{[#3 ---#2]}}
\newcommand{\todo}[1]{\textcolor{red}{[TODO: #1]}}
\else
\newcommand{\authornote}[3]{}
\newcommand{\todo}[1]{}
\fi

\newcommand{\wss}[1]{\authornote{magenta}{SS}{#1}} 
\newcommand{\plt}[1]{\authornote{cyan}{TPLT}{#1}} %For explanation of the template
\newcommand{\an}[1]{\authornote{cyan}{Author}{#1}}

%% Common Parts

\renewcommand{\progname}{
    ProgName}
\renewcommand{\authname}{
    Team \#, Team Name\\
    Student 1 name\\
    Student 2 name\\
    Student 3 name\\
    Student 4 name}
\usepackage{hyperref}
    \hypersetup{colorlinks=true, linkcolor=blue, citecolor=blue, filecolor=blue,
                urlcolor=blue, unicode=false}
    \urlstyle{same}
                                


\begin{document}

\maketitle

\begin{table}[hp]
    \caption{Revision History} \label{TblRevisionHistory}
        \begin{tabularx}{\textwidth}{llX}
        \toprule
        \textbf{Date} & \textbf{Developer(s)} & \textbf{Change}\\
        \midrule
            Sept. 16, 2025  & Changhao Wu           & Initial Draft                                         \\
            Sept. 20, 2025  & Changhao Wu           & Updated Problem Statement and Goals document          \\
            Sept. 20, 2025  & Changhao Wu           & Completed Reflection section                          \\
            Sept. 22, 2025  & Changhao Wu           & Final reviewed Problem Statement and Goals document   \\
            ...             & ...                   & ...                                                   \\
        \bottomrule
        \end{tabularx}
\end{table}

\section{Problem Statement}

% \wss{You should check your problem statement with the
% \href{https://github.com/smiths/capTemplate/blob/main/docs/Checklists/ProbState-Checklist.pdf}
% {problem statement checklist}.} 

% \wss{You can change the section headings, as long as you include the required
% information.}


\subsection{Problem}

    On modern dairy farms, many routine tasks require farmers to quickly locate and 
    separate specific cows from the herd. Examples include providing medical treatment 
    to a sick cow, monitoring the recovery of an injured animal, collecting milk samples 
    for quality testing, administering vaccines, or tracking cows with unusual feeding 
    or reproductive behaviors.  

    \setlength{\parskip}{0.6em} 

    Currently, locating specific animals is highly time-consuming. Farmers often 
    have to walk through an entire pen, visually check ear tag numbers one by one, 
    and manually identify the right cow. This process is not only slow and 
    labor-intensive, but also stressful for both workers and animals.

    \setlength{\parskip}{0.6em} 

    The problem is compounded by the fact that cows are large, mobile herd animals. 
    In a crowded barn or pasture, they are constantly moving, blocking each other 
    from view, or clustering together, making manual identification even more 
    difficult. Under these conditions, locating a single cow can take far longer 
    than planned and disrupt other farm operations. 


\subsection{Inputs and Outputs}


\begin{itemize}
    \item \textbf{Inputs}
          \begin{itemize}
                \item Visual data streams from on-farm environments.
                \item Supplementary annotated datasets for training and evaluation.
          \end{itemize}

    \item \textbf{Outputs}
          \begin{itemize}
              \item Information that distinguishes individual cows.
              \item Consistent identification of cows across time.
              \item Visual outputs annotated with each cow's predicted identity
          \end{itemize}
\end{itemize}


% \wss{Characterize the problem in terms of ``high level'' inputs and outputs.  
% Use abstraction so that you can avoid details.}


\subsection{Stakeholders}

\subsubsection*{\color{blue}{Direct Stakeholders}}
\begin{enumerate}
    \item\textbf{Cattleytics Inc. (Industry Partner)}: The primary stakeholder 
          and end user of the system. They benefit directly from improved efficiency 
          in locating, identifying, and monitoring individual cows for treatment, 
          feeding management, and general herd health.
    \end{enumerate}


\subsubsection*{\color{blue}{Indirect Stakeholders}}
\begin{enumerate}
    \item \textbf{Veterinarians}: By enabling faster and more reliable identification of target animals, 
            veterinarians can administer treatments and health checks more efficiently, improving overall herd health management.
    \item \textbf{Other Dairy Farmers and Ranchers}: Since the system is designed with relatively low deployment cost, it can be adopted by other farms to reduce 
            labor demands, increase operational efficiency, and ultimately improve profitabilit.
    \item \textbf{Animal Welfare and Protection Organizations}: Better tracking and monitoring ensure that cows receive 
            timely care, reducing stress and discomfort. These outcomes align with the goals of organizations advocating 
            for animal well-being and ethical treatment in agriculture.
\end{enumerate}


\subsection{Environment}

\subsubsection*{\color{blue}{Hardware Environment}}
\begin{itemize}
    \item Deployment will take place in dairy farm environments such as barns, feeding areas, or open pens, 
          where lighting, shadows, and occlusion from other animals may affect visibility.
    \item Input devices will consist of fixed surveillance cameras installed in barns or portable cameras (e.g., smartphones).
    \item Development and training will primarily be conducted on a GPU-enabled workstation 
      (e.g., a desktop equipped with an NVIDIA RTX 4060 or higher-class GPU).
    \item For later testing, the system may also be evaluated on lightweight hardware (e.g., NVIDIA Jetson) 
          to explore potential edge deployment, but the primary focus is on farm-level camera input rather than embedded control.
\end{itemize}


\subsubsection*{\color{blue}{Software Environment}}
\begin{itemize}
    \item Operating System: Ubuntu Linux (WSL2 on Windows will also be used for development).
    \item Programming Language: Python.
    \item Core Libraries and Frameworks: PyTorch, Ultralytics YOLOv8, OpenCV, scikit-learn, and ByteTrack.
    \item Development Tools: Git (for version control), GitHub (for remote collaboration and repository hosting), 
      GitHub Projects (for task management), GitHub Actions (for CI/CD), and VS Code as the primary IDE.
    \item Testing and Evaluation Tools: pytest for unit testing, flake8/black for linting and formatting, 
          and TensorBoard/Matplotlib for model evaluation and visualization.
\end{itemize}

% \wss{Hardware and Software Environment}


\section{Goals}
The primary goal of this project is to design and implement a computer vision system that can:
    \begin{enumerate}
        \item Accurately locate cows within visual inputs (images or video), ensuring that the presence of animals can be detected without manual inspection.
        \item For video data, maintain continuous tracking of individual cows across frames, ensuring that the same animal is consistently recognized over time.
        \item Provide a practical application scenario: given the information of a target cow (e.g., name or identifier), the system highlights this cow within the input data, eliminating the need for manual searching.
    \end{enumerate}

\section{Stretch Goals}
    \begin{itemize}
        \item Enable deployment on portable or embedded platforms, making the system accessible directly on-farm without requiring high-end computing infrastructure.
        \item Broaden applicability by extending recognition capabilities beyond cows to other livestock or farm management contexts.
        \item Deliver a farmer-oriented interface that visualizes animal activity and identity records, and allows data export to support decision-making and reporting.
    \end{itemize}


\section{Extras}
    \subsection{Performance Report:}
        A comprehensive performance report will be created to document the system's 
        accuracy, speed, and reliability. This will include quantitative evaluation of detection, tracking, 
        and identification models under different conditions (e.g., lighting, occlusion, herd density). 
        The report will provide insights into system limitations and suggest areas for future improvement.
    \subsection{User Manual:}
        A user-friendly manual will be prepared to guide non-technical users, such as farmers 
        and dairy workers, through system setup and operation. The manual will include step-by-step installation 
        instructions, examples of typical use cases, troubleshooting tips, and recommendations for best practices 
        when deploying the system on a farm.



% \wss{For CAS 741: State whether the project is a research project. This
% designation, with the approval (or request) of the instructor, can be modified
% over the course of the term.}

% \wss{For SE Capstone: List your extras.  Potential extras include usability
% testing, code walkthroughs, user documentation, formal proof, GenderMag
% personas, Design Thinking, etc.  (The full list is on the course outline and in
% Lecture 02.) Normally the number of extras will be two.  Approval of the extras
% will be part of the discussion with the instructor for approving the project.
% The extras, with the approval (or request) of the instructor, can be modified
% over the course of the term.}

\newpage{}

\section*{Appendix --- Reflection}

% \wss{Not required for CAS 741}

The purpose of reflection questions is to give you a chance to assess your own
learning and that of your group as a whole, and to find ways to improve in the
future. Reflection is an important part of the learning process.  Reflection is
also an essential component of a successful software development process.  

Reflections are most interesting and useful when they're honest, even if the
stories they tell are imperfect. You will be marked based on your depth of
thought and analysis, and not based on the content of the reflections
themselves. Thus, for full marks we encourage you to answer openly and honestly
and to avoid simply writing ``what you think the evaluator wants to hear.''

Please answer the following questions.  Some questions can be answered on the
team level, but where appropriate, each team member should write their own
response:


\begin{enumerate}

    \setlength{\parskip}{1.6em} 
    
    \item What went well while writing this deliverable?
    
    One positive outcome was that our team was able to refine the level of detail in our 
    problem statement and goals. Initially, we included too many technical specifics, which 
    risked restricting our future implementation choices. Under the guidance of the TA, 
    we revised these sections to be more abstract yet still measurable, ensuring they provide 
    direction without prematurely narrowing our design space. In addition, we carried out 
    early tests on potential external components that the project might depend on. This gave 
    us a baseline understanding of the tools and libraries available and allowed us to set 
    realistic goals that stretch our abilities without being unachievable.

    \item What pain points did you experience during this deliverable, and how
    did you resolve them?

    A major challenge was the relatively low and inconsistent engagement from some team 
    members, likely due to heavy workloads in other courses this term. This limited the 
    amount of sustained progress we could make week to week. As individuals, we recognize 
    that we need better time planning to avoid leaving tasks until just before deadlines. 
    Such last-minute work does not leave sufficient time to polish the deliverables to 
    a high standard. Another difficulty was our limited experience with disciplined use 
    of version control. While we managed to use Git and GitHub for collaboration, our 
    workflow lacked consistency, which could make the project harder to sustain in the 
    long run. To resolve this, we intend to adopt clearer commit conventions, branch organization, 
    and more frequent pushes so that our repository remains well-structured. 
    In addition, we plan to protect the \texttt{main} branch by requiring that all 
    changes go through a pull request, and that every pull request must be reviewed 
    and approved before merging. This process will help us maintain code quality 
    and accountability throughout the project.


    \item How did you and your team adjust the scope of your goals to ensure
    they are suitable for a Capstone project (not overly ambitious but also of
    appropriate complexity for a senior design project)?

    When defining our goals, we initially considered ambitious objectives such as pursuing 
    both the current project and an additional module for lameness detection in cows. 
    However, after further investigation, we realized that this direction would be far too 
    complex given the resources and circumstances currently available to us. Because lameness 
    detection has high commercial value, there are very few publicly available projects or 
    datasets we could reference. Building such a system would likely require us to collect data 
    from scratch, for example by visiting farms, recording videos, and learning how to reliably 
    label lameness under veterinary guidance. Given the scale of this work and the heavy 
    course workload that most of our team members are already carrying this term, it became 
    clear that attempting both projects in parallel would not be feasible. 

    After reflection, we scaled our scope down to focus on three core capabilities: detecting 
    cows, maintaining continuous tracking in video, and identifying individuals. Stretch goals 
    were then defined for deployment on embedded devices and broader applicability to other 
    livestock. Importantly, the scope we chose remains within a reasonable range of our current 
    technical capacity, challenging us to go slightly beyond our comfort zone without being 
    unrealistic. This balance ensures the project is both feasible and sufficiently challenging 
    to be worthy of a Capstone project.


\end{enumerate}  

\end{document}